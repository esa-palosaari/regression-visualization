\documentclass{article}
\usepackage[utf8]{inputenc}
\usepackage[finnish]{babel}
\usepackage{tabto}
\usepackage{hyperref}

\title{OS2-projektidokumentaatio \\ Regressiomallin visualisointi}
\author{Esa Palosaari \\ 
        618573, TiK, vk 2017}
\date{23.4.2019}

\begin{document}

\maketitle

\section{Yleiskuvaus}

    Tehtävänannon ja suunnitelman mukaisesti olen rakentanut ohjelman, joka sovittaa 
	regressiomallin datajoukkoon ja visualisoi tulokset. Kyseessä on vaikean tason työ. 
	Tämä tarkoittaa annettujen kriteerien mukaisesti, että käyttäjä voi ladata datan 
	valitsemastaan tiedostosta käyttäen kahta tiedostomuotoa (CSV ja XML). Ohjelman 
	käsittelemät tiedostot sisältävät (x, y) -koordinaattipareja desimaalilukuina 
	niin, että desimaalierottimena on piste. Ohjelmaa voi pyytää sovittamaan joko 
	suoran tai paraabelin muotoisen regressiomallin. Ohjelma piirtää kuvaajan 
	alkuperäisistä datapisteistä ja niihin sovitetusta mallista. Käyttäjä pystyy 
	säätämään kuvaajan asetuksia. Toteutus on laajennettavissa eli uusien ominaisuuksien 
	lisääminen ei ole vaikeaa tai vaadi koko ohjelman kirjoittamista uudelleen. 
	Erityisesti uusia uusia tiedostomuotoja, regressiomalleja ja käyttöliittymiä on 
	helppo lisätä. Tiedostonlukijasta ja regressiomallista on abstraktit luokat, 
	joista voi tehdä uusia konkreettisia aliluokkia. Abstraktit luokat määrittelevät 
	muiden rajapinnat ohjelman muille osille siten, että esimerkiksi uuden 
	regressiomallityypin regressikäyrien piirtämistä tai kaavan kirjoittamista 
	varten ei tarvitse muuttaa graafisia luokkia ollenkaan.  

\section{Käyttöohje}
    \subsection{Ohjelman "kääntäminen" komentoriviltä ajettavaksi}
    Ohjelman komentorivikäyttöliittymän JAR-tiedosto voidaan tehdä esimerkiksi Eclipsessä:
    \begin{itemize}
        \item Klikkaa oikealla näppäimellä pakettia \texttt{regressionViz}
        \item Valitse "Export..."
        \item Avaa Java-kansio
        \item Valitse "JAR file" ja klikkaa "Next"
        \item Valitse JAR-tiedostolle haluamasi paikka ja nimi 
        \item Klikkaa "Finish". 
    \end{itemize}

    Javan versio 8 olisi oltava asennettu. Eclipsessä tulisi olla Scalan versio 2.12.3. 
	Paketin \texttt{regressionViz} Build Path:ssä tulisi olla scala-xml\_2.12-1.0.x.jar 
	-tiedosto (se on liitetty mukaan gitiin, libs-hakemistoon). Ohjelman ajamista varten 
	tulisi komentorivin polussa olla Scalan versio 2.12.3 (ohjeet version asentamiseksi 
	esimerkiksi Ubuntulle: \\ \url{https://gist.github.com/malzzz/15639fc36d069490b7709be2f3a4d522}). 
	Käyttä-milläni yliopiston Ubuntu-koneilla Scalan versio komentorivillä oli 2.11.8, 
	Eclipsessä Scala oli 2.12.2. Scalan päivittäminen ei onnistu ainakaan helposti ilman 
	root-oikeuksia eikä ohjelma näytä toimivan oikein Scalan versiolla 2.11. Eclipsessä 
	komentorivitoimintoja voi toteuttaa seuraavalla tavalla:

    \begin{itemize}
        \item Klikkaa CLIApp.scalaa oikealla napilla ja valitse \texttt{Run As}
        \item Valitse \texttt{Run Configurations...}
        \item Kirjoita kohtaan \texttt{Main class:} \texttt{regressionViz.CLIApp}
        \item Välilehdelle \texttt{Arguments} kohtaan \texttt{Program arguments:} voi 
		kirjoittaa ohjelman optioita alla olevien ohjeiden mukaan. 
        \item Kohtaan \texttt{Program arguments:} voi myös kirjoittaa 
		\texttt{\$\{string\_prompt\}}, jolloin Eclipse antaa ikkunan ohjelman 
		optioiden kirjoittamiseen joka kerta kun ohjelman käynnistää.
    \end{itemize}
    
    \subsection{Ohjelman käyttäminen komentoriviltä}
    Komentorivin kautta ohjelma käynnistetään komennolla: \\
    
    \texttt{scala -cp "\$\{CLASSPATH\}:\$\{SCALA\_HOME\}/lib/scala-library.jar:} \\
    \texttt{NameOfTheJARExported.jar" \, regressionViz.CLIApp [options]}
    \\
    
    \texttt{NameOfTheJARExported.jar} on aiemmin valitsemasi nimi JAR-tiedostolle. 
    
        Ohjelman ohjeet saa näkyviin kun sen käynnistää ilman mitään optioita: \\
    
    \texttt{scala -cp "\$\{CLASSPATH\}:\$\{SCALA\_HOME\}/lib/scala-library.jar:} \\
    \texttt{NameOfTheJARExported.jar" \, regressionViz.CLIApp}
    \\
        
    Ohjeet kertovat mitä ja miten parametreja ohjelmalle voi käynnistysoptioiden 
	avulla antaa. Tässä käynnistysoptiot ja niiden selitykset: 
    
    \begin{itemize}
        \item \texttt{--d} Datatiedoston polku ja tiedostonnimi. Vaaditaan ohjelman 
		toimimiseksi. Nimen on päätyttävä joko .csv tai .xml. Kohdassa 6 kuvataan 
		datatiedostojen hyväksyttävät muodot.
        \item \texttt{--o} Polku ja nimi tallennettavalle PNG-kuvatiedostolle. Vaaditaan.
        \item \texttt{--varx} Ensimmäisen, selittävän muuttujan nimi. Tämä optio 
		vaaditaan vain XML-tiedoston käytön yhteydessä, jolloin sen on oltava sama 
		kuin XML-tiedostossa olevan muuttujan nimi. CSV-tiedostojen kanssa se on 
		valinnainen. Nimeä käytetään visualisaatiossa.
        \item \texttt{--vary} Toisen, selitettävän muttujan nimi. Vaaditaan 
		XML-tiedoston käytön yhteydessä, jolloin se  on oltava sama kuin 
		XML-tiedostossa olevan selitettävän muuttujan nimi.
        \item \texttt{--sizex} Kuvan vaakasuora koko pikseleinä. Jos optiota 
		ei anneta, käyttää ohjelma oletusarvoa on 500.
        \item \texttt{--sizey} Kuvan pystysuora koko pikseleinä. Jos optiota ei 
		anneta, käyttää ohjelma oletusarvoa on 500.        
        \item \texttt{--xmax} Vaakasuoran akselin suurin arvo. Oletuksena on 
		suurin ensimmäisen muuttujan arvo + yksi akselilla käytettävän suuruusluokan yksikkö.
        \item \texttt{--xmin} Vaakasuoran akselin pienin arvo. Oletuksena on 
		pienin ensimmäisen muuttujan arvo - yksi akselilla käytettävän suuruusluokan yksikkö.
        \item \texttt{--ymax} Pystysuoran akselin suurin arvo. Oletuksena on 
		suurin toisen muuttujan tai regressiokäyrän arvo suurimmalla ensimmäisen 
		muuttujan arvolla + yksi akselilla käytettävän suuruusluokan yksikkö.
        \item \texttt{--ymin} Pystysuoran akselin pienin arvo. Oletuksena on pienin 
		toisen muuttujan tai regressiokäyrän arvo pienimmällä ensimmäisen muuttujan 
		arvolla - yksi akselilla käytettävän suuruusluokan yksikkö.
        \item \texttt{--pR} Datapisteiden RBG-värin Red-arvo (0-255). Oletuksena 255.
        \item \texttt{--pB} Datapisteiden RBG-värin Blue-arvo (0-255). Oletuksena 0.
        \item \texttt{--pG} Datapisteiden RBG-värin Green-arvo (0-255). Oletuksena 0.
        \item \texttt{--cR} Regressiokäyrän RBG-värin Red-arvo (0-255). Oletuksena 0.
        \item \texttt{--cB} Regressiokäyrän RBG-värin Blue-arvo (0-255). Oletuksena 255.
        \item \texttt{--cG} Regressiokäyrän RBG-värin Green-arvo (0-255). Oletuksena 0.
        
    \end{itemize}
    
	Ohjelma osaa tällä hetkellä skaalata vaaka- ja pystyakselin asteikkoja ylöspäin 
	kertaluokasta \(10^0\),	mutta ei alaspäin. Jos kaikki muuttujan arvot ovat kertaluokkaa 
	\(10^{-1}\) tai pienempiä, datatiedostossa olevat arvot tulee korottaa muualla siten
	että muuttujan kertaluokka on vähintään \(10^0\).
	
	Tällä hetkellä ohjelma käyttää sekä pysty- että vaaka-akselin mitta-asteikkona samaa
	pikselimäärää. Eli jos vaaka-akselilla etäisyyteen 1.0 mahtuu 10 pikseliä, myös pysty-
	akselilla etäisyyteen 1.0 mahtuu 10 pikseliä. Tämä auttanee tulkitsemaan kuvaajia
	monissa kohdissa, mutta toisissa taas vaikeuttaa. Se myös rajoittaa, miten paljon
	akseleilta voi jättää piirtämättä, mikä voi johtaa rumiin kuviin.

\section{Ohjelman rakenne}

    \subsection{Ohjelman luokat ja niiden tehtävät}
    Ohjelman luokat jakautuvat tehtävien mukaan datan lukemiseen tiedostosta 
	(\texttt{DataReader, CSVReader, XMLReader}), datan liittyvään informaation säilyt-
	tämiseen (\texttt{Data}), regressiomallin sovittamiseen (\texttt{Model, OLSModel, QuadModel}) 
	ja sovittamisessa käytettäviin matriisioperaatiohin 
	(\texttt{ArrayMatrix}), visualisoinnin piirtä-miseen (\texttt{Drawing}, ja ohjelman 
	tehtävien suorittamiseen kaikkien luokkien avulla ja käyttäjän käskyjen mukaisesti 
	(\texttt{Engine, CLIApp}). Yksikkötestejä varten on oma luokkansa.
    
    Luokkarakenteessa on ollut keskeistä jatkokehityksen ja lisäominaisuuksien lisäämisen helppous. Sen vuoksi datan lukemisesta ja regressiomallin sovittamisesta on abstraktit luokat, joiden rajapinta-ominaisuuksia toteuttavia uusia luokkia on helppo lisätä ohjelmaan. Samoin käyttöliittymiä voi lisätä ja muuttaa ilman että koko ohjelmaa pitää kirjoittaa uusiksi, koska usea käyttöliittymä voi säilyttää olioita ja käyttää niiden toimintoja \texttt{Engine}-luokassa. Se voisi mahdollistaa myös useamman samanaikaisen näkymän ja käyttäjän ohjelmalle. Jos ei olisi ajatusta muuttaa tai lisätä ominaisuuksia ohjelmaan, voisi päästä vähemmällä kirjoittamisella poistamalla abstraktit luokat ja ottamalla toimintalogiikka ja olioiden säilytys osaksi käyttöliittymäluokkaa.
    
    \subsection{Luokkien keskeiset metodit}
    \begin{itemize}
        \item \texttt{DataReader readFile(filename: String): Option[Data]} Kaikki 
		DataReaderin perivät luokat (\texttt{CSVReader, XMLReader}) toteuttavat tämän 
		metodin, jolle annetaan tiedoston nimi, ja joka palauttaa Data-luokan olion 
		Optioniin käärittynä. Metodi siis lukee annetusta tiedostosta numeroarvot ja 
		tallentaa ne Data-olioon.
        \item \texttt{Data initializeDataset(newPoints, newName, newVarNames)} Metodin 
		avulla voi vaihtaa Data-olion datapisteitä, nimeä ja muuttujien nimiä. Näillä 
		Data-olion muuttujilla on oletusarvot, eikä kaikkia muuttujia tarvitse muuttaa 
		samalla kertaa. 
        \item \texttt{Data getPoints, getVarNames} palauttavat Optioniin käärittynä 
		Double-muotoiset datapisteet kaksiulotteisessa taulukossa ja Optioniin käärityn 
		ArrayBufferin muttujien nimiä.
        \item \texttt{Model fitData: Unit} sovittaa Model-luokan perivissä luokissa 
		(\texttt{OLSModel, QuadModel}) regressiomallin konstruktorissa annetun Data-olion 
		datapisteisiin. Sovitetun mallin, yhtälön ja yhtälön arvon voi saada omilla 
		getterimetodeilla.
		\item \texttt{ArrayMatrix multiplyMatrixAndVector} kertoo sopivan ulotteisen
		matriisin vektorilla vasemmalta ja palauttaa Double-tyyppisen vektorin.
		\item \texttt{ArrayMatrix multiplyMatrices} kertoo kaksi sopivan ulotteista
		matriisia keskenään ja palauttaa Double-tyyppisen matriisin.
		\item \texttt{ArrayMatrix invertMatrix} kääntää ja palauttaa neliömatriisin
		käänteismatriisin. Metodi olettaa että matriisi on kääntyvä.
		\item \texttt{Engine readData} lukee parametrina annetun datatiedoston tiedot
		käyttäen tiedoston päätteen mukaista lukijaa ja tallentaa osoittimen Data-olioon
		Engine-olion kokoelmamuuttujaan. Metodin parametreina annetut muuttujien nimet tallennetaan
		Data-olioihin.
		\item \texttt{Engine fitModel} sovittaa parametrina annettuun Data-olioon parametrina
		annetun tyyppisen regressiomallin. Viittaus Model-olioon lisätään Engine-olion
		kokoelmamuuttujaan.
		\item \texttt{Engine drawImage} luo annettujen parametrien mukaisen Drawing-olion
		ja lisää viittauksen siihen Engine-olion kokelmamuuttujaan. Ainoa vaadittu parametri
		on viittaus Model-olioon. Muut parametrit voivat olla None-tyyppejä.
		\item \texttt{Engine saveImage} tallentaa parametrina annetun Drawing-olion mukaisen
		kuvan PNG-formaatissa parametrina annetun nimiseen tiedostoon.
		\item \texttt{CLIApp main} parsii ohjelman kutsussa annetut optiot ja kutsuu Engine-olion
		metodeja annettujen parametrien ja arvojen mukaan.
		
    \end{itemize}
    
    

\section{Algoritmit}
    
    Pienimmmän neliösumman menetelmässä tavallisen regressioyhtälön ratkaisuun käytettiin 
	normaaliyhtälön kaavaa (Ordinary Least Squares, Wikipedia). Kaavalla laskemiseen kuului matriisien ja vektorien kertolaskuja ja transpooseja. Vaikein laskentatehtävä oli käänteismatriisin laskeminen, joka tapahtui Gauss-Jordan -menetelmällä (ks. Viitteet).
	
	Regressioyhtälön ja käänteismatriisin olisi voinut laskea muilla menetelmillä. Käytin
	tässä yksinkertaisimpia oppimisen ja ajankäytön vuoksi.

\section{Tietorakenteet}

	Ohjelmassa käytettävät tietorakenteet ovat joko yksi- tai kaksiulotteisia Arrayta ja 
	ArrayBuffereita. Yksi- ja kaksiulotteiset Arrayt ovat datapisteiden varastointia ja 
	käsittelyä varten silloin kun niiden koko tiedetään. Arrayt ovat nopeita ja yleisesti
	käytettyjä rakenteita varsinkin vektoreiden ja matriisien esittämiseen ja manipulointiin.
	ArrayBuffereita käytän silloin kun kokoelman koko muuttuu tai saattaa muuttua. Arrayn
	kokoa on hankalampi ja tehottomampi muuttaa kuin ArrayBufferin. Käytän lähinnä 
	muuttuvatilaisia rakenteita, koska en vielä hallitse funktionaalista tyyliä tarpeeksi
	hyvin. Javamaisempi tyyli (muuttuvatilaisia, mutta mahdollisesti yksityisiä rakenteita,
	joihin pääsee gettereillä ja settereillä) tuntuu vielä helpommalta eikä liian ongelmalliselta.
	
	Olisi varmaan voinut käyttää funktionaalisempaa tapaa muuttumattomine rakenteineen.
	Olisi myös voinut pärjätä yksiulotteisilla Arraylla, joiden rivit olisi voinut laskea
	moduloa käyttäen, mutta kaksiulotteisten taulukoiden hieman korkeampi abstraktitaso auttoi
	nyt ajattelemaan paremmin. Yksiulotteisilla taulukoilla voinee tiristää jotain tehokkuutta
	jatkossa jos haluaa.
    

\section{Tiedostot ja verkossa oleva tieto}

    Ohjelmalla käyttäjä voi sovittaa kahden muuttujan regressiomallin ja tallentaa datan ja 
	regressiomallin visualisaation PNG-tiedostoon. Muuttujien arvot annetaan tiedostoissa, 
	jotka ovat joko CSV (Comma Separated Variables) tai XML (Extensible Markup Language) 
	-muotoisia. CSV-tiedostoissa numeroiden desimaalierottimena on piste ja numerot erotetaan 
	toisistaan pilkulla. Eri muuttujien arvot ovat omissa sarakkeissa. Numerot voivat olla 
	tiedostossa esimerkiksi seuraavalla tavalla: \\
    
    \texttt{1.2, -3.43}
    
    \texttt{3.123, 2.3423}
    
    \texttt{4, 3.0} \\
    
    XML-tiedostoissa muttujien \texttt{PE} ja \texttt{MarketCap} arvot voidaan esittää esimerkiksi seuraavalla tavalla: \\
    
    \texttt{
    <?xml version="1.0"?> \\
    <symbols> \\
\tab{   <symbol ticker="Cisco" \,> \\
     \tab  \tab{     <PE>21.65</PE>} \\
      \tab \tab{     <MarketCap>271.18</MarketCap>} \\
        \tab{</symbol>} \\
    \tab{<symbol ticker="Sandisk"\,> \\
       \tab \tab{    <PE>23.71</PE>} \\
        \tab \tab{    <MarketCap>15.53873</MarketCap>} \\
        </symbol>} \\
        <symbol ticker="Oracle"\,> \\
    \tab   \tab{     <PE>17.87</PE>} \\
    \tab    \tab{    <MarketCap>225.61</MarketCap> }\\
        </symbol> \\
        <symbol ticker="Red Hat"\,> \\
    \tab    \tab{    <PE>65.41</PE>} \\
    \tab    \tab{    <MarketCap>32.24</MarketCap>} \\
        </symbol> }\\
    </symbols>\\
    }
    
    Datatiedostoissa ei saa olla puuttuvia tietoja (eli tyhjiä kohtia) tai datana muuta kuin numeroita. 
	CSV-tiedoston ensimmäisen rivin mahdolliset muuttujien nimet on poistettava ennen tiedoston lukemista 
	ohjelmalla. Ohjelma tunnistaa datatiedoston muodon sen päätteestä (.csv tai .xml). Jos tiedoston 
	lukemisessa on ongelmia, kannattaa tarkistaa ettei datatiedostoissa ole mukana ylimääräisiä "näkymättömiä"
	merkkejä kuten tabulaattoreita. Tämä vaikutti aiheuttavan ongelmia CSV-muotoisessa testitiedostossa
	yliopiston koneella.
    
\section{Testaus}

    Ohjelmaa testattiin heti alusta yksikkötestein. Yksikkötestit kirjoitettiin ennen varsinaista 
	ohjelmakoodia, ja testien tuli ensin ilmoittaa virheestä. \texttt{Drawing}-luokan joitain
	metodeita testattiin yksikkötestein, 
	mutta pääasiallinen testaustapa oli vain katsoa kuvia
	joita luokka tuotti ja verrata niitä haluttuihin kuvaajiin. Komentorivityökalua ja Engine-
	luokkaa testattiin myös lähinnä kokeilemalla niitä. 
	
	Suunnitelmassa mainituista testeistä ohjelma läpäisee sen, että regressioyhtälö
	palauttaa oikeat arvot virherajojen puitteissa kun syötteenä on ennalta tunnettuja
	datoja. En ole keskittynyt testeissä algoritmin virhealttiuteen enkä siihen, miten
	täydellinen lineaarinen riippuvuus vaikuttaa. Datan lukeminen näyttää toimivan
	suunnitellusti, mutta yliopiston koneessa oli jokin yllättävä ongelma, minkä syy
	ei täysin selvinnyt. Puuttuvien tietojen käsittely ei näyttänyt toimivan niin
	kuin ajattelin, minkä takia jätin sen ajanpuutteen vuoksi lopulta vain pois.
	
	Komentorivikäyttöliittymän omilla silmillä testaamiset ohjelma näyttää läpäisevän.

\section{Ohjelman tunnetut puutteet ja viat}

	\begin{itemize}
		\item Akseliyksiköt ovat samat mikä voi olla ongelma, kun kertaluokat ovat erit tai pitää saada
		\item pystyakselin yksiköt eivät lähde nollasta isoilla numeroilla
		\item pysty- ja vaaka-akselin automaattinen sovitus tuottaa välillä liialta näyttävää ylimääräistä
	  tilaa
		\item puuttuvan tiedon käsittely eli sellaisten rivien poisto, joista toisesta muuttujasta
	  puuttui tieto, ei toiminut
		\item skaalaus alaspäin ei toimi, pienin numero akseleilla on yksi. Tämä liittynee kokonaislukuihin.
		\item numeroiden mahtuminen akseleille, kun mennään suureen määrään numeroita per akseli
	\end{itemize}
	
\section{3 parasta ja 3 heikointa kohtaa}     

	\subsection{3 parasta kohtaa}
	\begin{itemize}
		\item Regressiomallin sovittamisen toimivuus. Regressiomallin estimointi normaaliyhtälöllä
			  vaikuttaa toimivan täsmälleen kuten pitääkin. Tuntui melkein taikuudelta,
			  että ohjelma osasi sovittaa oikean toisen asteen käyrän vain viiden datapisteen 
			  perusteella. Normaaliyhtälöä on myös helppo laajentaa korkeamman asteen polynomeihin.
			  Ohjelmaan ei olisi kovin vaikeata lisätä ominaisuutta, jossa sille annettaisiin vain
			  luku nollasta n:ään regressioyhtälön asteesta ja ohjelma sovittaisi halutun asteen
			  polynomin.
		\item Kuvaajan oikeellisuus. Vaikka kuvaajan visuaalisten apukeinojen, erityisesti akseleiden
			  lukujen automatiikassa on parannettavaa, kuvaajat näyttävät olevan "matemaattisesti"
			  tosia. Toisin sanoen, pisteet, origosta lähtevät apuakselit ja regressiokäyrä ovat
			  teknisesti oikeissa suhteissa toisiinsa (mittausvirheen ja pikseleiden rajoissa). 
		 \item Ohjelman rakenne ja laajennettavuus. Ohjelman rakenne on mielestäni selkeä: Luokat vastaavat
                   kukin yhdestä pääasiasta ja ne sisältävät tehtävään tarvittavan datan ja menetelmät.
                   Luokkien keskittyminen omaan asiaansa ja abstraktien luokkien käyttö tekevät ohjelman
                   laajentamisesta helppoa. Uuden käyttöliittymän lisääminen ei vaadi koskemista
                   muuhun ohjelmaan. Erilaisia regressiomalleja tai luettavia tiedostomuotoja voi lisätä
                   mielivaltaisen määrän kirjoittamalla niille abstraktin luokan mukaiset toteutukset
                   ja niitä vastaavat kutsut Engine-luokkaan.
                  
	\end{itemize}
	
	\subsection{3 heikointa kohtaa}
	\begin{itemize}
		\item Vaaka- ja pystyakselin koon automaattinen määritys voi tuottaa epäesteettisiä eli rumia
			  lopputuloksia. Automatiikkaa voisi hienosäätää paremmaksi näpräilemällä lisää pikselien sekä
			  akseleiden ja yksiköiden skaalojen kanssa. Suuren osan epäesteettisyyksistä
			  voi korjata ohjelmalle annetuilla parametreilla, mutta parametreilla ei voi tällä hetkellä
			  muuttaa kaikkia kohtia. Esimerkiksi kummallekin akselille saman suuruinen numeerinen
			  etäisyys tarkoittaa saman suuruista pikselimäärää. Omat yksiköt eri akseleille tuottavat
			  uusia ongelmia, mutta niitä voisi ratkoa ja antaa käyttäjälle mahdollisuuksia säätää
			  akseleiden yksiköiden pikselimäärää.
		\item Käyttöliittymä. Tätä kirjoittaessa komentorivityökalukäyttöliittymä ei ole kovin kummoinen. Se
                      sopii parhaiten tilanteisiin tai työ, joissa käytetään muita vastaavia komentorivityökaluja.
		\item Käynnistämisen hankaluus. Komentorivityökalun käynnistäminen eri koneilla, joilla on erilaisia
                      Scalan versioita on epävarmaa ja hankalaa.
	\end{itemize}

\section{Poikkeamat suunnitelmasta, toteutunut työjärjestys ja aikataulu}

        Viikkojen 7 ja 8 aikana tein ja demosin suunnitelmat kuten olin suunnitellutkin. Gitin olin
        saanut toimimaan 27.2. eli viikolla yhdeksän kuten olin suunnitellut. Aloitin Data-
        luokan kirjoittamisesta, jota tein yksikkötestien kanssa 27.2.-7.3. En ollut laittanut
        suunnitelmaan yhtään aikaa Data-luokan tekemiselle, mutta siinä menikin viikko. Model ja
        OLSModel-luokkia tein 7.3.-15.3. Tässä vaiheessa olin viikon myöhässä aikataulusta.
        Tein jossain välissä projektidokumentaation pohjan kuten olin suunnitellut. Suunnitelmastani
        puuttui kokonaan Drawing-luokka, joka osoittautui yhdestä tärkeimmistä ja vaikeimmista
        ohjelman osista. Tein sitä 18.3.-21.3, 4.4., 11.-12.4., 15.-16.4. Drawing-luokka jäi
        suunnitelmasta pois varmaankin sen vuoksi, etten ollut ratkaissut suunnitelmaa tehdessäni
        miten toteuttaisin varsinaisen piirtämisen.
        Aloitin datan lukemisen testaamisen ja kirjoittamisen 21.-26.3. Tällöin oli siis viikko 13,
        kun alkuperäisen suunnitelman mukaan olisi pitänyt olla viikko 11. Komentorivin
        (\texttt{CLIApp}) ja Engine-luokan ohjelmointi tapahtui noin 26.3.-11.4. (viikot 13-15).
        Alkuperäisen suunnitelman mukaan tällöin olisi pitänyt tehdä graafinen käyttöliittymä
        ja viimeistellä koodia. Lisäsin toisen asteen regressiopolynomin ohjelmaan 13.4.
        (\texttt{QuadModel}). XMLReader-luokan tein 14.4.

        Toteuttamisjärjestys toteutui pääpiirteissään, mutta eritysesti Data- ja Drawing-luokkiin
        meni huomattavan paljon aikaa eivätkä ne olleet alkuperäisessä suunnitelmassa mukana.
        Myös ohjelman saaminen toimimaan komentoriviltä vei enemmän aikaa kuin mitä odotin
        aiempien C ja C++ -kokemusteni perusteella.

        Alussa tein yleensä noin 10 tuntia viikossa, tenttiviikon aikoihin vähemmän ja lopussa
        enemmän. Tein yhteensä vähemmän kuin suunnitelin, koska muut opinnot ja työt söivät aikaa.
        Varasin työlle puskuriaikaa, koska arvelin että yllättäviä asioita tapahtuu. Näin kävikin.
        Jälkikäteen arvioituna käytin alussa liikaa aikaa toisarvoisten ominaisuuksien kuten
        puuttuvien tietojen käsittelyn hiomiseen.
        
\section{Kokonaisarvio lopputuloksesta}    

        Ohjelma täyttää vaatimukset: se lukee kahdenlaisia datatiedostoja, sovittaa niihin
        kahdenlaisia regressiomalleja, ja lopuksi tuottaa malleista visualisoinnin, jonka
        yksityiskohtiin käyttäjä voi vaikuttaa ja jonka voi tallentaa tiedostoon. Ohjelmaan
        voi helposti lisätä uusia ominaisuuksia tai muuttaa luokkien toteutuksia ilman että
        koko ohjelmaa pitäisi kirjoittaa uudelleen. Uusien regressiomallien ja tiedostomuotojen
        lisäksi erilaisten käyttöliittymien lisääminen pitäisi olla kivutonta. Käyttölogiikkaa
        (\texttt{Engine}) olen rakentanut niin, että käyttöliittymissä on esimerkiksi
        mahdollista käyttää undo- ja redo-toimintoja.

        Huonot puolet liittyvät tällä hetkellä visualisoinnin yksityiskohtiin ja niin sanottuun
        käyttöliittymään. Eri skaalat pysty- ja vaaka-akselilla johtavat helposti rumiin
        kuviin vaikka kuvat olisivatkin sinänsä paikkansa pitäviä. Asiaan voisi vaikuttaa
        mahdollistamalla eri pikselimäärät samoille yksiköille pysty- ja vaaka-akselilla.

        Akseleiden skaalaus kokonaisluvuista alaspäin eli numeron kymmenen negatiiviisiin potensseihin
        ei myöskään tällä hetkellä toimi, minkä kiertämiseksi hyvin pieniä lukuja sisältäviä
        datatiedostoja tulisi käsitellä etukäteen ennen niiden syöttämistä ohjelmalle. Asiaa
        voisi lähteä korjaamaan esimerkiksi vaihtamalla akseleiden yksiköt kokonaisluvuista
        liukuluvuiksi.

        Akseleiden pituudet voisivat vastata paremmin annettuja päätepisteitä regressiokäyrälle.
        Ohjelmalle voisi tehdä graafisen käyttöliittymän. Minua kiinnostaisi lisätä ohjelmaan
        mahdollisuus saada datapisteiden tiedot näkyviin kun niiden päälle vie hiiren kursorin.
        Eniten minua kuitenkin kiinnostaa taustalla olevan matematiikan parantaminen. Käytetty
        normaaliyhtälö on yksinkertaisin menetelmä regressikertoimien laskemiseen, mutta se on
        myös virhealtis. Aion ohjelmoida muita algoritmeja. Aion myös ohjelmoida p-arvojen
        laskemisen regressioyhtälön kertoimille, sekä epävarmuuden kuten luottamusvälien
        visualisointeja.

        Olen tyytyväinen tietorakenteisiin ja luokkajakoon. Ohjelma on helposti laajennettavissa
        ja muutettavissa.

        Jos aloittaisin projektin nyt uudelleen alusta, en käyttäisi niin paljoa yksityiskohtien
        tai ei-tarpeellisten ominaisuuksien hiomiseen kuin mitä nyt tein alussa. Yrittäisin
        sen sijaan toteuttaa ensin toimivan rungon, johon voisi lisätä ominaisuuksia käytettävissä
        olevan ajan mukaan. Siihen suuntaan aloin projekin loppua kohti mennä. 

\section{Viitteet}    

        Algorithm to round to the next order of magnitude in R. Stack Overflow.
        [Viitattu 23.4.2019]. \\
        \url{https://stackoverflow.com/questions/7906996/algorithm-to-round-to-the-next-order-of-magnitude-in-r} \\

        Best way to parse command-line parameters? Stack Overflow. [Viitattu 23.4.2019]. \\ 
        \url{https://stackoverflow.com/questions/2315912/best-way-to-parse-command-line-parameters} \\

        Drawing images. Otfried Cheong. [Viitattu 23.4.2019]. \\
        \url{http://otfried.org/scala/drawing.html} \\        
        
        How to process a CSV file in Scala. Alvin Alexander. [Viitattu 23.4.2019]. \\
        \url{https://alvinalexander.com/scala/csv-file-how-to-process-open-read-parse-in-scala} \\

        How to rotate text with Graphics2D in Java? Stack Overflow. [Viitattu 23.4.2019]. \\
        \url{https://stackoverflow.com/questions/10083913/how-to-rotate-text-with-graphics2d-in-java} \\

        Inverse Matrix by Gauss Jordan Elimination. Java Programming Lab Code. [Viitattu 23.4.2019]. \\
        \url{http://cljavacode.blogspot.com/2017/06/inverse-matrix-by-gauss-jordan.html} \\

        Ohjelmointistudio 2. Aalto-yliopisto. [Viitattu 23.4.2019]. \\
        \url{https://plus.cs.hut.fi/studio\_2/k2019/} \\

        Ordinary Least Squares. Wikipedia. [Viitattu 23.4.2019]. \\
        \url{https://en.wikipedia.org/wiki/Ordinary\_least\_squares} \\

        Working With XML in Scala. Mahesh Chand. [Viitattu 23.4.2019]. \\
        \url{https://dzone.com/articles/working-with-xml-in-scala} \\

\section{Liitteet}    

        \subsection{Liite 1: Lähdekoodi}
        Lähdekoodi on saatavilla kokonaisuudessaan osoitteesta: \\
        \url{https://version.aalto.fi/gitlab/palosae2/regression-visualization}

        \subsection{Liite 2: Ajoesimerkkejä}
        Ohjelma lukee CSV-tiedoston ja käyttää oletusregressiomallia (ensimmäisen asteen yhtälö)
        ja visualisoinnin oletuasarvoja:\\

  \texttt{scala -cp "\$\{CLASSPATH\}:\$\{SCALA\_HOME\}/lib/scala-library.jar:} \\
  \texttt{NameOfTheJARExported.jar" \, regressionViz.CLIApp --d datatiedosto.csv} \\
  \texttt{--o kuvatiedosto.png} \\

  Ohjelma lukee XML-tiedostosta muuttujat varName1 ja varName2 ja sovittaa toisen asteen yhtälön datapisteisiin: \\

  \texttt{scala -cp "\$\{CLASSPATH\}:\$\{SCALA\_HOME\}/lib/scala-library.jar:} \\
  \texttt{NameOfTheJARExported.jar" \, regressionViz.CLIApp --d datatiedosto.xml} \\
  \texttt{--o kuvatiedosto.png --modeltype quad --varx varName1 --vary varName2}\\

  Ohjelma lukee CSV-tiedoston; sovittaa toisen asteen yhtälön; antaa nimet muuttujille; tekee kuvaajan,
  jonka koko on 600x600 pikseliä; rajoittaa kuvaajan vaaka-akselin välille (-1, 5); ja värjää
  datapisteet keltaisiksi. Regressiokäyrän väri ja y-akselin väli ovat oletuksen mukaisia: \\

  \texttt{scala -cp "\$\{CLASSPATH\}:\$\{SCALA\_HOME\}/lib/scala-library.jar:} \\
  \texttt{NameOfTheJARExported.jar" \, regressionViz.CLIApp --d datatiedosto.csv} \\
  \texttt{--o kuvatiedosto.png --modeltype quad --varx X --vary Y --sizex 600 --sizey 600}
  \texttt{--xmin -1 --xmax 5 --pR 255 --pB 215 --pG 20}
      
        
\end{document}
